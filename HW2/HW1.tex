
\documentclass[12pt]{article}

\usepackage{algorithm}
\usepackage{algpseudocode}
\usepackage{enumitem}
\usepackage[body={7in,9.5in},centering]{geometry}
\usepackage{fancyhdr}
\usepackage{times}
\usepackage{tikz}
\usepackage{hyperref}

\lhead{}
\rhead{}
\renewcommand{\headrulewidth}{3pt}
\lfoot{}
\cfoot{}
\rfoot{}
\pagestyle{empty}

\newcommand*\circled[1]{\tikz[baseline=(char.base)]{
            \node[shape=circle,draw,inner sep=2pt] (char) {#1};}}


\begin{document}
\begin{center}
  \Large
  CIS 675 (Fall 2018) Disclosure Sheet 
\end{center} 
\vspace*{2em}

\noindent
\textbf{\Large Name: \underline{ Wentan Bai }} 


\noindent 
\begin{minipage}[t]{1.0\linewidth}

\begin{minipage}[t]{0.25\linewidth}
\textbf{\Large
  HW \# \underline{ 2 }
} 

\end{minipage} \vspace*{3ex}




\begin{minipage}[t]{.8in}
  \textbf{\circled{Yes} \quad No}
\end{minipage}
\qquad 
\begin{minipage}[t]{5.5in}
  Did you consult with anyone on parts of this assignment, including other students, TAs, or the instructor? 
\end{minipage}
\vspace*{1ex}

\begin{minipage}[t]{.8in}
  \textbf{\circled{Yes} \quad No}
\end{minipage}
\qquad 
\begin{minipage}[t]{5.5in}
  Did you consult an outside source (such as an Internet forum or a
  book other than the course textbook) on parts of this assignment? 
\end{minipage}
\vspace*{1ex}

\noindent
  If you answered \textbf{Yes} to one or more questions, please give the details here: \vspace*{5ex} \\
  I consulted all questions with teaching assistant, Arash Sahebolamri, to confirm my understanding of all questions are correct. 
  I also consult question 3, question 5, question 6 and extra question with another student, Wentian Bai. For question 3, question 5 and question 6, we discussed our ideas and I finish my algorithms independently. 
  For the extra question, we discussed our understandings and I write my explanation independently.  \vspace*{5ex} \\

  For the question 1, I view some mathematical formulas online. \\ \href{url}{http://tutorial.math.lamar.edu/Classes/CalcI/LHospitalsRule.aspx}


\vfill
\end{minipage}



\vspace*{40ex}

By submitting this sheet through my Blackboard account, I assert that the information on this sheet is true.

%\vfill

\hfill {\tiny This disclosure sheet was based on one originally designed
  by
  Profs. Royer and Older.}
  
 \pagebreak
\large Question 1: \vspace{5mm} \\
\normalsize 
\begin{algorithm}
\begin{algorithmic}
\Function{find}{$A$, left, right}
    \State \textbf{while} $right == 1$:
    \State \hspace{0.8cm}  left = right
    \State \hspace{0.8cm}  $right = right * 2$
    \State \textbf{}$midpoint = left + (right - left) / 2$
    \State \textbf{if} $A[midpoint] == 1$ and $A[midpoint + 1] == $ "error message":
    \State \hspace{0.8cm}  \textbf{return} $midpoint$
    \State \textbf{else if} $A[midpoint] == 1$:
    \State \hspace{0.8cm}  \textbf{return find}($A$, midpoint, right)
    \State \textbf{else}:
    \State \hspace{0.8cm}  \textbf{return find}($A$, left, midpoint)
\EndFunction
\end{algorithmic}
\end{algorithm}\vspace{3mm}\\


\pagebreak
\large Question 1:\\
\normalsize 
\begin{enumerate}[label=(\alph*)]
  \item $f(n) = n$ \hspace{1cm} $g(n) = n^2 - n$ \vspace{1mm} \\
    $L = \lim_{n\to\infty} \frac{f(n)}{g(n)} = \lim_{n\to\infty} \frac{n}{n^2 - n} = \lim_{n\to\infty} \frac{n}{n(n-1)}$ \vspace{1mm} \\
    $= \lim_{n\to\infty} \frac{1}{n - 1} = 1$ \vspace{1mm} \\
    If $0 < L < \infty,$ $f(n) = \theta(g(n))$ and $f(n) = O(g(n)) $
  \item $f(n) = \log_{}{5n} + 2$ \hspace{1cm} $g(n) = \log_{}{2n} + 5$ \vspace{1mm} \\
    In BIg O notation, n is consider as infinite, so we should only compare $log_{}{5n}$ with $log_{}{2n}$. \vspace{1mm} \\
    $L = \lim_{n\to\infty} \frac{f(n)}{g(n)} = \lim_{n\to\infty} \frac{\log_{}{5n} + 2}{\log_{}{2n} + 5} 
    = \lim_{n\to\infty} \frac{\log_{}{5n}}{\log_{}{2n}}$ \vspace{2mm} \\
    Based on L'Hopital Rule, \vspace{1mm} \\
    $f_1(n) = \lim_{n\to\infty} \log_{}{5n} = \infty $ \vspace{1mm} \\
    $g_1(n) = \lim_{n\to\infty} \log_{}{2n} = \infty $. \vspace{1mm} \\
    Therefore, $\lim_{n\to\infty} \frac{\log_{}{5n}}{\log_{}{2n}} = \lim_{n\to\infty} \frac{f_1'(n)}{g_1'(n)}$ \vspace{1mm} \\
    $f_1'(n) = \frac{d}{dn}\log_{}{5n} = \frac{1}{n}$ \vspace{1mm} \\
    $g_1'(n) = \frac{d}{dn}\log_{}{2n} = \frac{1}{n}$ \vspace{1mm} \\
    Thence, $L = \lim_{n\to\infty} \frac{\log_{}{5n}}{\log_{}{2n}} = \lim_{n\to\infty} \frac{f_1'(n)}{g_1'(n)} = 1$ \vspace{1mm} \\
    Since, $\lim_{n\to\infty} \frac{\log_{}{2n + 5}}{\log_{}{5n + 2}}$ can be proved as the same way and get the same answer.
    If $0 < L < \infty,$ $f(n) = \theta(g(n))$ so $f(n) = O(g(n)) $ and $ g(n) = \theta(f(n))$ so $g(n) = O(f(n)) $.
  \item $f(n) = 10\log_{}{n}$ \hspace{1cm} $g(n) = \log_{}{n^4}$ \vspace{1mm} \\
    $L = \lim_{n\to\infty} \frac{f(n)}{g(n)} = \lim_{n\to\infty} \frac{10\log_{}{n}}{\log_{}{n^4}} = \lim_{n\to\infty} \frac{10\log_{}{n}}{4\log_{}{n}} $ \vspace{1mm} \\
    $L = \lim_{n\to\infty} \frac{\log{}{n}}{\log{}{n}} = 1 $\vspace{2mm} \vspace{1mm} \\
    Since, $\lim_{n\to\infty} \frac{g(n)}{f(n)}$ can be proved as the same way and get the same answer. Therefore, $f(n) = O(g(n)) $ and $g(n) = O(f(n)) $ since $0 < L < \infty$. 
  \item $f(n) = 100n + (\log_{}{n})^2$ \hspace{1cm} $g(n) = 100n + \log_{}{n}$ \vspace{1mm} \\
    Firstly, I need to compare $100n$ with $\log_{}{n}$. \vspace{1mm} \\
    $\lim_{n\to\infty} \frac{100n}{\log_{}{n}} = 100\cdot\lim_{n\to\infty} \frac{n}{\log_{}{n}} = 100\cdot\lim_{n\to\infty} \frac{1}{\frac{1}{n}} = 100\cdot\infty = \infty$ \vspace{1mm} \\
    This means $\log_{}{n} = o(100n)$ and when $n\rightarrow\infty$, we should only consider about 100n for g(n). \vspace{1mm} \\
    Then, I need to compare $100n$ with $(\log_{}{n})^2$. \vspace{1mm} \\
    $\lim_{n\to\infty} \frac{(\log_{}{n})^2}{100n} = \frac{1}{100} \cdot \lim_{n\to\infty}(\frac{\log{}{n}}{\sqrt{n}})^2 = \frac{1}{100} \cdot \lim_{n\to\infty}(\frac{\ln{n}}{\sqrt{n}})^2$  \vspace{1mm} \\
    $=\frac{1}{100} \cdot \lim_{n\to\infty}(\frac{\frac{1}{x}}{\frac{1}{2\sqrt{n}}})^2 = \frac{1}{100} \cdot \lim_{n\to\infty}(\frac{2}{\sqrt{n}})^2 = 0$ \vspace{1mm} \\
    This means $(\log_{}{n})^2 = o(100n)$ and when $n\rightarrow\infty$, we should only consider about 100n for f(n). \vspace{1mm} \\
    Therefore, $L = \lim_{n\to\infty} \frac{f(n)}{g(n)} = \lim_{n\to\infty} \frac{100n}{100n} = 1  $\vspace{2mm} \vspace{1mm} \\
    Since, $\lim_{n\to\infty} \frac{g(n)}{f(n)}$ can be proved as the same way and get the same answer. Therefore, $f(n) = O(g(n)) $ and $g(n) = O(f(n)) $ since $0 < L < \infty$.
  \item $f(n) = n!$ \hspace{1cm} $g(n) = 2^n$ \vspace{1mm} \\
    Based on Big-O Notation: $g(n) = O(f(n))$ iff there exists a constant c and a fixed $n_0$ such that for all $n > n_0$, $g(n) \leq cf(n)$. \vspace{1mm} \\
    When $n \geq 4$, $g(n) < f(n)$. However, whatever constant c is, there is not a fixed $n_0$ to satisfy the Big-O notation. Therefore, $g(n) = O(f(n))$.
\end{enumerate}


\pagebreak
\large Question 2:\\
\normalsize 
\begin{enumerate}
\item Recurrence Relation: \\
$T(n) = 3 \cdot T(\frac{n}{3}) + O(n^0)$.
\item Running time:\\
Based on Master Method, $T(n) = a \cdot T(\frac{n}{b}) + O(n^d)$,\\
where a = 3, b = 3 and d = 0.\\
The ${\log_3 3}$ = 1 $>$ 0. \\
Therefore the result is O($n{\log_3 3}$) = O($n$).
\end{enumerate}

\pagebreak
\noindent
\large Question 3: \vspace{5mm} \\
\normalsize 
At beginning, we can put a thin book or a normal book or a wide book. Then for each selection, we also need to choose to put a thin book or a normal book or a wide book. Therefore, the pseudo code is: \\
\begin{algorithm}
\begin{algorithmic}
\Function{ways}{int $N$}
    \State \textbf{if} $N == 0$:
    \State \hspace{0.8cm}  \textbf{return} $0$
    \State \textbf{else if} $N == 1$:
    \State \hspace{0.8cm}  \textbf{return} $1$
    \State \textbf{else if} $N == 2$:
    \State \hspace{0.8cm}  \textbf{return} $2$
    \State \textbf{else if} $N == 3$:
    \State \hspace{0.8cm}  \textbf{return} $4$
    \State \textbf{else}:
    \State \hspace{0.8cm} \textbf{return ways}($N-1$) $+$ \textbf{ways}($N-2$) $+$ \textbf{ways}($N-3$)
\EndFunction
\end{algorithmic}
\end{algorithm}\\
The Recurrence Relation is: \\
T(0) = $C_0$; \\
T(1) = $C_1$;\\
T(2) = $C_2$;\\ 
T(3) = $C_3$ \\
T(N) = T(N-1) + T(N-2) + T(N-3) \\



\pagebreak
\large Question 4: \vspace{5mm}\\
\normalsize 
\underline{Claim:} if $f(x) = o(g(x))$, then we must have that $f(x) = O(g(x))$. \vspace{2mm}\\
\underline{Proof:} By contradiction, support $f(x) = o(g(x))$ and $f(x) \neq O(g(x))$ for some $f(x)$ and $g(x)$. Based on Big-O Notation, $f(n) \neq O(g(n))$ means there does \textbf{not} exist a constant c and a fixed $n_0$ such that for all $n > n_0$ , $f(n) \leq cg(n)$. However, in Little-o Notation, $f(x) = o(g(x))$ means, for \textbf{every} c $>$ 0, there exists a fixed $n_0$ such that for all $n > n_0$, $f(n) \leq cg(n)$. There is contradiction, so my supposition is incorrect. Therefore, original proposition is true.


\pagebreak
\noindent
\large Question 5: \vspace{5mm}\\
\normalsize 
\begin{algorithm}
\begin{algorithmic}
\Function{find}{$A$, left, right}
    \State \textbf{while} $right == 1$:
    \State \hspace{0.8cm}  left = right
    \State \hspace{0.8cm}  $right = right * 2$
    \State \textbf{}$midpoint = left + (right - left) / 2$
    \State \textbf{if} $A[midpoint] == 1$ and $A[midpoint + 1] == $ "error message":
    \State \hspace{0.8cm}  \textbf{return} $midpoint$
    \State \textbf{else if} $A[midpoint] == 1$:
    \State \hspace{0.8cm}  \textbf{return find}($A$, midpoint, right)
    \State \textbf{else}:
    \State \hspace{0.8cm}  \textbf{return find}($A$, left, midpoint)
\EndFunction
\end{algorithmic}
\end{algorithm}\vspace{3mm}\\
I use two pointers, left and right, to traverse the given array. Initialize left = 0 and right = 1. Then use a while loop to double left and right each time until right returns an error message. When right first returns an error message, it means right is out off the bounds of array A. And since left always equals to the previous right, the n will be in the range from left to right. Then we can use binary search to find n. Declare a variable midpoint which is the middle point of left and right. If the midpoint returns 1 and its next returns error message, the midpoint is the bounds and return it. If midpoint still returns 1 and its next is not error message, the bounds is in range from midpoint to right. Then ecursively call function and limit the range from midpoint to right. Otherwise, the midpoint exceeds the bound of array and we recursively call function and limit the range from left to midpoint. \vspace{5mm}\\
For the while loop, since it double each time and the first iteration takes the longest time, the worest running time of while loop is $log_2{n}$. For the recurrence part, I shrink half of current range each time, so using the Master Method, the b is 2 and a is 1. The loop part only works when we need to find bounds, so the d is 0. The Recursion Relation is $T(n) = T(\frac{n}{2}) + O(n^0) + O(\log_{2}{n})$ $log_b{a} = 0 = d$. Therefore, the running time is $O(n^d\cdot \log_{}{n})$ = $O(\log_{}{n})$.


\pagebreak
\noindent
\large Question 6:\\
\normalsize 
\begin{algorithm}
\begin{algorithmic}
\Function{check}{L, K, start, sum result}:
    \State \textbf{if} result == False:
    \State \hspace{0.8cm}  \textbf{return} result
    \State \textbf{if} $sum < (len(K))^2$:
    \State \hspace{0.8cm}  result = False
    \State \hspace{0.8cm}  \textbf{return} result
    \State \textbf{for} i in range(start, len(L)):
    \State \hspace{0.8cm} K.add(L[i])
    \State \hspace{0.8cm} result \&\& check(L, K, i + 1, sum+L[i], result)
    \State \textbf{return} result
\EndFunction \\
\end{algorithmic}
\end{algorithm}\vspace{3mm}\\

For this question, I design a fucntion check() to find and check all subsets $K \subseteq L$. At beginning, initials K as an empty set, start = 0, sum = 0 and result = True. The variable sum is the sum of elements in current $K$.  If there exists a subset K that $sum < |K|^2$, then the claim is false and returns the result as False. Otherwise, check other possible subsets. The $L[start]$ is the first element of current $K$. Recursively find all possible subsets with first element $L[start]$. After that, move start to next and recursively find all possible subsets with first element $L[start+1]$. For each recursion, let result \&\& with return of recursion. As long as there exists a result that is Flase, the final result will be False. Otherwise, the claim is True. \vspace{3mm}\\
The running time is $O(2^n)$ since there are $2^n$ subsets which we need to check.



\pagebreak
\noindent
\large \textbf{Extra}:\\
\normalsize 


If $f(x) = O(g(x))$, it must be the case that $\lim_{x\to\infty} f(x)/g(x) = c$, for some finite $c$. Let $L = \lim_{x\to\infty} f(x)/g(x)$. If $L = \infty$, then $g(x) = o(f(x))$ and $f(x) = \omega(g(x))$. Based on Little-o-notation, if $g(x) = o(f(x)) $, then for every $c > 0$, there exists a fixed $x_0$ such that for all $x > x_0$, $g(n) \leq cf(x)$. Therefore, there does \textbf{not} exists a constant c and a fixed $n_0$ such that for all $n > n_0$, $f(n) \leq cg(n)$. So based on Big-O-notation, $f(n) \neq O(g(n))$. Hence, when $\lim_{x\to\infty} f(x)/g(x) = c$, for some finite $c$, $f(x) = O(g(x))$.




  
 \end{document} 
























