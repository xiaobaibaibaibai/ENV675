
\documentclass[12pt]{article}

\usepackage{algorithm}
\usepackage{algpseudocode}
\usepackage{enumitem}
\usepackage[body={7in,9.5in},centering]{geometry}
\usepackage{fancyhdr}
\usepackage{times}
\usepackage{tikz}
\usepackage{hyperref}
\usepackage{graphicx}

\lhead{}
\rhead{}
\renewcommand{\headrulewidth}{3pt}
\lfoot{}
\cfoot{}
\rfoot{}
\pagestyle{empty}

\newcommand*\circled[1]{\tikz[baseline=(char.base)]{
            \node[shape=circle,draw,inner sep=2pt] (char) {#1};}}


\begin{document}
\begin{center}
  \Large
  CIS 675 (Fall 2018) Disclosure Sheet 
\end{center} 
\vspace*{2em}

\noindent
\textbf{\Large Name: \underline{ Wentan Bai }} 


\noindent 
\begin{minipage}[t]{1.0\linewidth}

\begin{minipage}[t]{0.25\linewidth}
\textbf{\Large
  HW \# \underline{ 4 }
} 

\end{minipage} \vspace*{3ex}




\begin{minipage}[t]{.8in}
  \textbf{\circled{Yes} \quad No}
\end{minipage}
\qquad 
\begin{minipage}[t]{5.5in}
  Did you consult with anyone on parts of this assignment, including other students, TAs, or the instructor? 
\end{minipage}
\vspace*{1ex}

\begin{minipage}[t]{.8in}
  \textbf{\circled{Yes} \quad No}
\end{minipage}
\qquad 
\begin{minipage}[t]{5.5in}
  Did you consult an outside source (such as an Internet forum or a
  book other than the course textbook) on parts of this assignment? 
\end{minipage}
\vspace*{1ex}

\noindent
  If you answered \textbf{Yes} to one or more questions, please give the details here: \vspace*{5ex} \par
  I consulted question 5 with teaching assistant, Siddhartha Roy Nandi, to confirm my understanding. 
  I also consult all questions and extra question with another student, Wentian Bai. For all questions, we discussed our ideas and I finish my algorithms independently. 

 

\vfill
\end{minipage}



\vspace*{40ex}

By submitting this sheet through my Blackboard account, I assert that the information on this sheet is true.

%\vfill

\hfill {\tiny This disclosure sheet was based on one originally designed
  by
  Profs. Royer and Older.}


\pagebreak
\noindent
\large Question 1: \vspace{5mm} \par
\normalsize 
My algorithm is.
\begin{itemize}
  \item	Place one guard at $l_1 + 1$ location, where 1 represent the 1 unit of distance. 
  \item move along the hallway until there is a painting $l_m$ , which is not protected. Place one guard at $l_m + 1$ location.
  \item Repeat the second step until all paintings are protected. 
\end{itemize}

\begin{algorithm}
\begin{algorithmic}
\State guard $\leftarrow$ $l_1 + 1$
\State num $\leftarrow$ 1
\State \hspace{0.4cm} \textbf{For} i = 2 to k :
\State \hspace{0.8cm} \textbf{If} guard $<$ $l_i$
\State \hspace{1.2cm} \textbf{} guard $\leftarrow$ $l_i + 1$
\State \hspace{1.2cm} \textbf{} num $\leftarrow$ $num + 1$
\end{algorithmic}
\end{algorithm}
\noindent \\
\textbf{running time:} \par
The running time of Modified Dijkstra’s shortest-path algorithm is the same as the time of original algorithm because the change part costs the same time as replaced part. \par
Thus, the final running time is double of running time of Dijkstra’s shortest-path algorithm. 
The final time is $O((|V| + |E|)\log_{}{|V|})$



\pagebreak
\noindent
\large Question 2: \vspace{5mm} \par
\normalsize 
My goal is to maximize the amount of payment. Thus, my idea is trying to assign the high-paying jobs to the slot just before their deadline. My algorithm is:
\begin{itemize}
  \item	Sort all jobs by their payments in decreasing order. 
	Denote the elements of sorted list as $J_1, J_2, ..., J_m$, and for each job $J_i$, $i \leftarrow 1...m $, $p_i \leq p_{i+1}$ .
  \item	Schedule $J_1$ to time slot from time $d_1-1$ to time $d_1$. 
  \item Iterate through the list of remaining jobs in order, and at each step $i$, check whether we can schedule current job $J_i$ from time $d_i-1$ to time $d_i$. \\
	If this time slot is already scheduled with other job, traverse the schedule forward from time $d_i-1$ and find whether there is a empty slot to schedule current job. \\ 
	If there is not a empty slot after traversing, we will abandon this job.
  \item After going through list, the schedule is expected result. 
\end{itemize}

\begin{algorithm}
\begin{algorithmic}
\State \textbf{Sort} all jobs by their payments in decreasing order.
\State schedule[m] \textbf{ //} m-size array
\State schedule[$d_1$] $\leftarrow$ $J_1$
\State \textbf{For} i = 2 to m :
\State \hspace{0.4cm} \textbf{If} schedule[$d_i$] \textbf{is} empty
\State \hspace{0.8cm} \textbf{}schedule[$d_i$] $\leftarrow J_i$
\State \hspace{0.4cm} \textbf{Else :}
\State \hspace{0.8cm} \textbf{For} k = $d_i-1$ to 1
\State \hspace{1.2cm} \textbf{If} schedule[k] \textbf{is} empty
\State \hspace{1.6cm} \textbf{} schedule[k] $\leftarrow J_i$
\State \textbf{Return} schedule
\end{algorithmic}
\end{algorithm}
\noindent \\
\textbf{running time:} \par
The sorting all jobs uses $O(n\log_{}{n})$. 
When we iterate through sorted jobs, for each job, we may need to traverse the schedule forward which uses $O(n)$ time.
Therefore, whole iteration need $O(n^2)$ time. \par
The final time is $O(n\log_{}{n}) + O(n^2) = O(n^2)$


\pagebreak
\noindent
\large Question 3: \vspace{5mm} \par
\normalsize 
\setlength{\baselineskip}{8mm}
Based on description of question, I only need to give a \textbf{reasonable greedy} algorithm. \par
For current city, I will find another city which is not visited and has the shortest distance between it and current city.
Then mark current city as visited and move to newly found city.
Repeat this step. \par
My algorithm will not always find the correct answer. For example: \\

1. In this graph, we begin at $C_0$. The shortest distance is $C_0 \rightarrow C_1$. Move to $C_1$ and mark $C_0$ as visited. \\
2. Then at $C_1$, the shortest distacne is $C_1 \rightarrow C_2$ without visited cities. Move to $C_2$ and mark $C_1$ as visited. \\
3. Then at $C_2$, the shortest distacne is $C_2 \rightarrow C_3$ without visited cities. Move to $C_3$ and mark $C_2$ as visited. \\
4. Then at $C_3$, the shortest distacne is $C_3 \rightarrow C_4$ without visited cities. Move to $C_4$ and mark $C_3$ as visited. \\
5. All cites visited and back to $C_0$. \par
The total distacne is $2+2+11+5 = 20$.
However, if we follow the path $C_0 \rightarrow C_2 \rightarrow C_1 \rightarrow C_3 \rightarrow C_0$, the total distance will be $3+2+3+5=13$.
Thus, my algorithm does works but does not find the correct answer. \\
\noindent \\
\textbf{running time:} \par
My algorithm visits all cities exactly once. Thus the time is $O(|V|)$, where $|V|$ means the number of all cities.


\pagebreak
\noindent
\large Question 4: \par
\normalsize 
\setlength{\baselineskip}{8mm}
Based on description of question, I think this procedure does not give me the correct shortest path from $s$ to $t$.
For example: \\
 \par
In this example, we parallely use Dijkstra’s algorithm for node $t$ and node $s$. 
Since the length $l(s, A)$ is shorter than $l(s, C)$ and $l(t, B)$ is shorter than $l(t, F)$, the $dist(A)$ will be smaller than $dist(C)$ and $dist(B)$ will be smaller than $dist(B)$.
Then there will be a overlap between node $A$ and node $B$ without finding the real shortest path. 
Based on description, $d_1 = 5 + 11$, $d_2 = 5$, $d_1 + d_2 = 21$. This is incorrect. \par
Thus, this algorithm does not find the correct answer. \\


\pagebreak
\noindent
\large Question 5: \\
\normalsize 
\setlength{\baselineskip}{8mm}
\noindent
\textbf{(1):} \par
Based on description of question, I think the directed graph G seems as below. \\
 \par
The infinite path $p$ is infinite sequence $s, a, b, c, d, b, c ,d, b, c, ..., b, c, d, e, f$, where "..." is infinite loop. 
The $Inf(p)$ will be a set $\{b, c, d\}$. \\
\\
\textbf{Claim: } If $p$ is an infinite path of $G$, then the $Inf(p)$ is a subset of a single strongly connected component of $G$. \\
\textbf{Proof: } By contradiction. Suppose $p$ is an infinite path of $G$, and the $Inf(p)$ is not a subset of a single strongly connected component of $G$. 
Based on description and my example, the vertices in an infinite path $p$ are visited infinitely often because there is a cycle. 
The $Inf(p)$ is the set of vertices that occur infinitely many times in p. Therefore, the vertices in this set are connected by some directed path. \par
The defination of SCC is: in a directed graph, SCC is a set of nodes such that there is a (directed) path between every pair in both directions.
Based on defination of SCC, for each set of a single strongly connected component of $G$, their nodes are connected by a path. 
Since the vertices in $Inf(p)$ are all connected, the $Inf(p)$ must be a subset of a single strongly connected component of G. 
There is a contradiction and my suppose is incorrect. The claim holds.



\pagebreak
\noindent
\large Question 5: \\
\normalsize 
\setlength{\baselineskip}{8mm}
\noindent
\textbf{(2):} \par
If graph $G$ has an infinite path, then there will be a cycle in $G$.
Since $G$ has a finite number of vertices, only in a cycle, some vertices of $G$ are able to be visited infinitely often. \par
Thus, my algorithm is to determine if $G$ has a cycle.
I use DFS on $G$. If depth-first search reveals a back edge of a directed graph $G$, $G$ has a cycle and there is a infinite path. 
If we explore all nodes and there does not exist a back edge, then $G$ does not has an infinite path.   \\
\\
\textbf{Claim: } If the DFS reveals a back edge, the graph has a cycle.\\
\textbf{Proof: } let $(c, s)$ be a back edge from node $c$ to node $s$. By definition of back edge, $s$ is an ancestor of $c$.
In the DFS tree, there is a path from $s$ to $c$. The path $s \rightarrow c$ and back edge $(c, s)$ form a cycle. The claim holds. \\
\\
\textbf{running time:} \par
My algorithm is the same as basic DFS. Thus my running time is $O(|V| + |E|)$.



\pagebreak
\large \textbf{Extra}:\\ \vspace{5mm}\par
\normalsize 
\setlength{\baselineskip}{8mm}
For each binary tree, the number of its edges $|E|$ equals to the number of its nodes minus one, $|E| = |V| - 1$. 
If the a binary tree has a perfect matching, the $|V|$ must be even. 
Since there is only one root node at the first level.
Thus, for a binary tree $T$, we need to check every node:
\begin{itemize}
\item For current node, if the number of all descendants in left is odd and the number of all descendants in right is even or zero, this partition has a perfect matching. 
\item For current node, if the number of all descendants in right is odd and the number of all descendants in left is even or zero, this partition has a perfect matching. 
\item If current node is a leaf, ignore it and return back to check its parent.
\item For current node, if the number of all descendants in right is and the number of all descendants in left are both even or odd, this whole tree does not have a perfect matching. 
\end{itemize}
I recursively go to deepest level and recusively back with check each node.

\textbf{running time:} \par
My algorithm is check all nodes one time. Thus my running time is $O(|V|)$.


\end{document} 







