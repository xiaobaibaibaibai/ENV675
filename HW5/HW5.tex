
\documentclass[12pt]{article}

\usepackage{algorithm}
\usepackage{algpseudocode}
\usepackage{enumitem}
\usepackage[body={7in,9.5in},centering]{geometry}
\usepackage{fancyhdr}
\usepackage{times}
\usepackage{tikz}
\usepackage{hyperref}
\usepackage{graphicx}

\lhead{}
\rhead{}
\renewcommand{\headrulewidth}{3pt}
\lfoot{}
\cfoot{}
\rfoot{}
\pagestyle{empty}

\newcommand*\circled[1]{\tikz[baseline=(char.base)]{
            \node[shape=circle,draw,inner sep=2pt] (char) {#1};}}


\begin{document}
\begin{center}
  \Large
  CIS 675 (Fall 2018) Disclosure Sheet 
\end{center} 
\vspace*{2em}

\noindent
\textbf{\Large Name: \underline{ Wentan Bai }} 


\noindent 
\begin{minipage}[t]{1.0\linewidth}

\begin{minipage}[t]{0.25\linewidth}
\textbf{\Large
  HW \# \underline{ 5 }
} 

\end{minipage} \vspace*{3ex}




\begin{minipage}[t]{.8in}
  \textbf{\circled{Yes} \quad No}
\end{minipage}
\qquad 
\begin{minipage}[t]{5.5in}
  Did you consult with anyone on parts of this assignment, including other students, TAs, or the instructor? 
\end{minipage}
\vspace*{1ex}

\begin{minipage}[t]{.8in}
  \textbf{\circled{Yes} \quad No}
\end{minipage}
\qquad 
\begin{minipage}[t]{5.5in}
  Did you consult an outside source (such as an Internet forum or a
  book other than the course textbook) on parts of this assignment? 
\end{minipage}
\vspace*{1ex}

\noindent
  If you answered \textbf{Yes} to one or more questions, please give the details here: \vspace*{5ex} \par
  I also consult all questions and extra question with another student, Wentian Bai. For all questions, we discussed our ideas and I finish my algorithms independently. 


\vfill
\end{minipage}



\vspace*{40ex}

By submitting this sheet through my Blackboard account, I assert that the information on this sheet is true.

%\vfill

\hfill {\tiny This disclosure sheet was based on one originally designed
  by
  Profs. Royer and Older.}


\pagebreak
\noindent
\large Question 1: \vspace{5mm} \par
\normalsize 
Suppose the given grid has $n$ numbers of rows and $m$ numbers of columns. 
\begin{itemize}
  \item On the Westernmost road, all T[i][1] = T[i+1][1] + $d_{i,1}$, 
	since we can only go straight north to the current position at the previous intersection.
  \item	On the Southernmost road, all T[n][j] = T[n][j-1] + $d_{n,j}$,
	since we can only go straight east to the current position at the previous intersection.
  \item Then for remaining intersections, we can go straight east or straigth north to the current position at the previous intersection.
	Therefore, we choose the shorter one, T[i][j] = $d_{i,j}$ + \textbf{min}(T[i+1][j], T[i][j-1])
\end{itemize}

\begin{algorithm}
\begin{algorithmic}
\State T[n][m]
\State T[n][1] = $d_{n,1}$
\State \textbf{For} i $\leftarrow$ n-1 to 1 \textbf{:}
\State \hspace{0.4cm} T[i][1] $\leftarrow$ T[i+1][1] + $d_{i,1}$
\State \textbf{For} j $\leftarrow$ 2 to m \textbf{:}
\State \hspace{0.4cm} T[n][j] $\leftarrow$ T[n][j-1] + $d_{n,j}$
\\
\State \textbf{For} i $\leftarrow$ n-1 to 1 \textbf{:}
\State \hspace{0.4cm} \textbf{For} j $\leftarrow$ 2 to m \textbf{:}
\State \hspace{0.8cm} T[i][j] = $d_{i,j}$ + \textbf{min}(T[i+1][j], T[i][j-1])
\\
\State \textbf{Return} T[1][m] 
\end{algorithmic}
\end{algorithm}
\noindent \\
\textbf{running time:} \par
We iterate through all intersections so the final time is $O(n^2)$.



\pagebreak
\noindent
\large Question 2: \vspace{5mm} \par
\normalsize 
This question is similar as Splitting a String.
I need to write a helper function to determine whether a particular string is a palindrome.

\begin{itemize}
  \item we check if a sub string from i to j is a palindrome. 
	If current sub string is a palidrome, we check if the previous sub string is a palidrome, which result is saved at Seq(i-j-1).
	If all requirements are satisified, save Seq(i) as True.
  \item The helper function is to determine whether a particular string is a palindrome. 
	We iterate from head and tail to center at the same time. 
	If there are a pair of characters which are different, this particular string is not a palindrome.
\end{itemize}

\begin{algorithm}
\begin{algorithmic}
\State Seq(i) = False
\State Seq(0) = True
\State \textbf{For} i = 1 to n \textbf{:}
\State \hspace{0.4cm} \textbf{For} j = 1 to i \textbf{:}
\State \hspace{0.8cm} \textbf{if} \textbf{helper}(Seq[i-j ... i]) == True and Seq(i-j-1) == True
\State \hspace{1.2cm} Seq(i) $\leftarrow$ True
\State \textbf{Return} Seq(n)
\end{algorithmic}
\end{algorithm}
Following is helper function: 
\begin{algorithm}
\begin{algorithmic}
\State def \textbf{helper}(Seq)
\State \hspace{0.4cm} \textbf{if} length(Seq) == 1 or 0
\State \hspace{0.8cm} \textbf{Return} False
\State \hspace{0.4cm} i = 0, j = length(Seq) - 1
\State \hspace{0.4cm} \textbf{While} i $\not=$ j and i $<$ j \textbf{:} 
\State \hspace{0.8cm} \textbf{if} Seq[i] $\not=$ Seq[j] \textbf{:}
\State \hspace{1.2cm} \textbf{Return} False
\State \hspace{0.8cm} i++, j- -
\State \textbf{Return} True
\end{algorithmic}
\end{algorithm}
\noindent \\
\textbf{running time:} \par 
When we iterate through all sub strings, at worst case we need $O(n^2)$.
The final time is $O(n^2)$


\pagebreak
\noindent
\large Question 3: \vspace{5mm} \par
\normalsize 
By hint, I modify the graph. 
In original graph, there are not capacities on the edges, so I will add capacities for each edges.

\begin{itemize}
  \item Assign $\infty$ or the sum of capacities of all nodes to the capacities of all edges from source node. 
  \item If the current node only points to one node and is only pointed by one node.
	then assign the same size of capacity on current node to the edge from current node. 
  \item If the current node $u$ points to multiple nodes or is pointed by multiple nodes, 
	then make node $u$ only connect with a new node $v$ which has the same capacity.
	Then assign the same size of capacity on node $u$ to the $Edge(u, v)$ 
	Connect node $v$ to the nodes which are original pointed by node $u$. 
	Assign the same size of capacity on node $u$ to the capacities of these new edges.  	

\end{itemize}
\par
Following is an example. The first one is original graph and the second is modified graph. \\
\includegraphics[width=12cm, height=4cm]{question3L} \\
\includegraphics[width=15cm, height=4cm]{question3R} 

\par
We can find a max flow using original Ford-Fulkerson algorithm on the modified graph.
\\

\textbf{running time:} \par
The time is the same as the time of original Ford-Fulkerson algorithm, $O(|V|\cdot|E|^2)$.



\pagebreak
\noindent
\large Question 4: \par
\normalsize 
My algorithm is: 

\begin{itemize}
  \item Let each animal be a node $a_i$. Then let each doctor be a node $D_j$. Create a source node $s$ and a sink node $t$. 
  \item Connet source node with every node $a_i$, and each $Edge(s, a_i)$ has a capacity $h_i$.
	Input flow is the time of the animals to be treated 
  \item If animal $a_i$ can be treated by doctor $D_j$, connect $a_i$ with $D_j$, and each $Edge(a_i, D_j)$ has a capacity $C_j$.
  \item Connect every doctor $D_j$ to sink node $t$, and each $Edge(D_j, t)$ has a capacity $C_j$. 
	This setp ensures each doctor $j$ works at most $C_j$ time. 

\end{itemize}

Following is an example.
\includegraphics[width=6cm, height=10cm, angle=90,origin=c]{question4}
\par
We can get the answer by using original Ford-Fulkerson algorithm on this graph.
\\

\textbf{running time:} \par
The time is the same as the time of original Ford-Fulkerson algorithm, $O(|V|\cdot|E|^2)$.




\pagebreak
\noindent
\large Question 5: \par
\normalsize 
My algorithm is:
\begin{itemize}
  \item	Let each paper be a node $p_j$ and each reviewer be a node $R_i$. Create a source node $s$ and a sink node $t$.
  \item Connet source node with every node $p_j$, and each $Edge(s, p_j)$ has a capacity $3$. 
  \item Connect each node $p_j$ with every node $R_i$ if paper $p_j$ is not submitted by reviewer $R_i$.  
	Each $Edge(p_j, R_i)$ has a capcity 1. 
  \item Connect every node $R_i$ with sink node, and each $Edge(s, p_i)$ has a capacity $m_i$.
  \item Run max flow algorithm. 
	However, in this algorithm, we must ensure that in the final graph all flows from source node to each node $p_j$ is 3, 
	which means that each node $p_j$ is reviewed by 3 times.
\end{itemize}

\noindent \\
\textbf{running time:} \par
The time is the same as the time of original Ford-Fulkerson algorithm, $O(|V|\cdot|E|^2)$.



\pagebreak
\large \textbf{Extra}:\\ \vspace{5mm}\par
\normalsize 

Create an array $dp[M][N]$: $dp[i][j]$ presents we need to choose $i$ numbers of fruits from $j$ numbers of boxes. \par
In initialization, if $m_1$ $\geq$ 1, assign $dp[1][1]$ to 1; else assign $dp[1][1]$ to 0. \par
For all $dp[1][2]$ to $dp[1][N]$, in each step $j$, if $m_j$ $\geq$ 1, then assign $dp[1][j]$ = $dp[1][j - 1] + 1$; else assign $dp[1][j] = dp[1][j - 1]$. 
In each step $j$, we need to choose one fruit from $j$ numbers of boxes. 
If $B_j$ is not empty, then there will be one more ways to choose.
Otherwise, there are still the $dp[1][j - 1]$ numbers of ways. \par
For all $dp[2][1]$ to $dp[M][1]$, in each step $i$, if $m_1$ $\geq$ $i$, then assign $dp[i][1] = 1$; else assign $dp[i][1] = 0$. 
In each step $i$, we need to choose $i$ numbers of fuits from $B_1$.
If $m_1$ is larger than required numbers of fruits $i$, then there is one type to choose; Otherwise, there is no fruit to choose. \par
Iterate from $i = 2$ to $M$, and for each i, iterate from $j = 2$ to $N$, 
and in each step $i$ $j$, assign $dp[i][j] = dp[i - 1][j] + dp[i][j - 1]$. \par
The final result is in $dp[M][N]$.


\begin{algorithm}
\begin{algorithmic}
\State dp[M][N]
\State dp[1][1] $\leftarrow$ 0
\\
\State \textbf{For} j = 2 to N :
\State \hspace{0.4cm} \textbf{If} $m_j$ $\geq$ 1 \textbf{:}
\State \hspace{0.8cm} $dp[1][j]$ $\leftarrow$ $dp[1][j - 1] + 1$
\State \hspace{0.4cm} \textbf{Else :}
\State \hspace{0.8cm} sum1[i] $dp[1][j]$ $\leftarrow$ $dp[1][j - 1]$
\\
\State \textbf{For} i = 2 to M :
\State \hspace{0.4cm} \textbf{If} $m_1$ $\geq$ $i$ \textbf{:}
\State \hspace{0.8cm} $dp[i][1]$ $\leftarrow$ 1
\State \hspace{0.4cm} \textbf{Else :}
\State \hspace{0.8cm} $dp[i][1]$ $\leftarrow$ 0
\\
\State \textbf{For} i = 2 to M :
\State \hspace{0.4cm} \textbf{For} j = 2 to N :
\State \hspace{0.8cm} $dp[i][j]$ $\leftarrow$ $dp[i - 1][j] + dp[i][j - 1]$
\\
\State \textbf{return} $dp[M][N]$
\end{algorithmic}
\end{algorithm}


\textbf{running time:} \par
Interation from 2 to M using M time, and for each step, interation from 2 to N using N time.
Therefore, the final time is $O(N M)$


\end{document} 







